\section{Set Theory}


%todo need to convert from handwriting to this
\subsection{Solution 1}

\begin{enumerate}
  \item \( A  \subseteq B\) and \( B \subseteq C \). Then let \( x \in  A  \implies x \in B \implies x \in  C\).
    Hence, for all \( x \in A, x \in C \). Therefore \( A \subseteq C. \)

    \item \( B \subseteq A \). This means \( x \in  B \implies x \in  A \).
      So, all elements of \( B \) are in \( A. \)
      Now, we show that \( A \cup B = A.\)
      Let \( x \in  A \cup B  \) then \( x \in  A  \) or \( x \in  B \).

      If \( x \in  A \) then we have nothing more to show. Otherwise if \( x \in  B \).
      Since \(  B  \) is a subset of \( A \). This means that \( x \in  A \) as well.
      Hence in either eventuality \( x \in  A \). So, for all \( x \in  A \cup  B \), \( x \in  A \).
      Hence \( A \cup  B \subseteq A \).

      It remains to show that \( A \subseteq A \cup  B \).
      Let \( x \in  A  \). then by definition of union \( x \in  A \cup  B \).
      
      Hence, for all \( x \in  A \), \( x \in  A \cup  B \). Therefore, 
      \( A \subseteq A \cup  B \).

      Hence we haveshown that \( A \subseteq A\cup  B \) and \( A \cup  B \subseteq A \).
      Therefore, \( A \cup  B = A \).

      \item \( B \subseteq A \), so \( x \in  B  \implies x \in  A\). Now,
        let \( x \in  B \cup  C \). then \( x \in  B \) or \( x \in  C \).

      If \( x \in  B \) then \( x \in  A \), hence \( x \in  A \cup  C \).

      Otherwise if \( x \in  C \). then \( x \in  A \cup  C \) by definition of union.

      So, in either eventuality \( x \in  A \cup  C \). 
      Now, we show that \( B \cap C \subseteq A \cap  C \).

      If \( x \in  B \cap C \), \( x \in  B \) and \( x \in  C \). Since, \( x \in  B \), \( x \in  A \). So, \( x \in  A \). and \( x \in  C \).
      Therefore, \( x \in A \) and \( x \in  C \). Therefore, \( x \in  A \cap C \).
\end{enumerate}

\subsection{Solution 2}

\begin{enumerate}
  \item Let \( x \in A \cap B \) then \( x \in  A \) and \( x \in  B \).
    Hence, \( x \in  A \cap B \) by definition. Therefore, \( A \cap  B \subseteq B \cap  A \).

    Let \( x \in  B \cap A \), then \( x \in  B \) and \( x \in  A \). Hence \( x \in  A \cap  B \).
    Therefore, \( B \cap A \subseteq A \cap  B \).

    Now we show that \( A \cup  B = B \cup  A \). Similar to above, trivial.

  \item Let \( x \in  A \cap \left(B \cap  C  \right) \). Then \( x \in  A \) and \( x \in B \cap C \implies x \in B \) and \( x \in  C. \) Hence
    \( x \in  A \) and \( x \in  \left( B \cap  C  \right) \). Therefore by definition \( x \in  A \cap \left( B \cap  C \right) \).
\end{enumerate}

\subsection{Solution 9}

  \begin{align*}
    \left(A + B  \right) + C &= \left( \left( A - B \right) \cup \left( B - A \right) \right) + C \\
    &= \left( \left( \left( A - B \right) \cup \left( B - A \right) \right) - C \right) \cup \left( C - \left( \left( A - B  \right) \cup \left( B- A \right) \right) \right)
  .\end{align*}

  Now we expand the RHS,
  \begin{align*}
    A + \left( B + C \right) &= A + \left( \left( B - C \right) \cup  \left( C - B \right) \right) \\
    &= \left( A - \left( \left( B - C \right) \cup  \left( C - B \right) \right) \right) \cup \left( \left( \left( B - C \right) \cup  \left( C - B \right) \right) - A \right)
  .\end{align*}

  Now, let \( x \in \left( A + B \right) + C \), then there are two cases 

  \begin{itemize}
   \item \( x  \) is an element of \( \left( \left( A - B \right) \cup \left( B - A \right) \right) - C \) . If this is the case then
     we know that \( x \in \left( A - B \right) \) or \( x \in  \left( B - A \right) \). However, in either case
     \( x \notin C \). We examine both cases,
     \begin{itemize}
      \item \( x \in \left( A - B \right) \) and \( x \notin C \): In this case we know that
        \( x \notin B \cup C \). So it is definitely not in a reduced version of this union which is
        \( \left( B - C \right) \cup  \left( C - B \right) \) as this is just removing further elements from
        \( B  \) and \( C \) before doing a union. Hence \( x \notin \left( B - C \right) \cup \left( C - B \right) \) but \( x \in A \). 
        Therefore, \( x \in A - \left( \left( B - C \right) \cup \left( C - B \right) \right)\) and hence 
        \( x \in  \left( A - \left( \left( B - C \right) \cup  \left( C - B \right) \right) \right) \cup \left( \left( \left( B - C \right) \cup  C - B \right) - A \right)\). Hence,
        in this case the subset relation \( \left( A + B \right) + C \subset A +  \left(   B + C\right) \) holds.

      \item \( x \in  \left( B - A \right) \) and \( x \notin C \). Hence, \( x \notin B \cap C \). So, it is in
        \( \left( B - C \right) \cup  \left( C - B \right) \) as this is the same as \( \left( B \cup C \right) - \left( B \cap C \right) \) by definition of \( B + C \). However, \( x \notin A \).
      Therefore, \( x \in  \left( \left( B - C \right) \cup \left( C - B \right)  \right)  - A \) and hence \( x \in \left( A - \left( \left( B - C \right) \cup \left( C - B \right) \right) \right) \cup \left( \left( \left( B - C \right) \cup \left( C - B \right) \right)  - A \right) \).
      Hence, in this case the subset relation \( \left( A + B \right) + C \subset A + \left( B + C \right) \) holds.

      \item not bothered rn come back to this later too tedious
    \end{itemize}
  \end{itemize}

\subsection{Solution 10}

\begin{enumerate}
 \item  Nope, transitivity is not guaranteed.

    \item Nope, transitivity is not satisfied.

\item     Yep, all three conditions satisfied, transitivity since uniqueness of father, the other two are trivial.


\item    Yep, all three conditions satisfied.  
 
\end{enumerate}

 

 \subsection{Solution 11}

 Not sure about this one.

 \subsection{Solution 12}

 The relation of concern is defined as follows. The set \( S \) of all integers
 and let \(  n > 1\) be a fixed integer. Define for \( a, b \in S \), \( a \sim  b \) if \( a - b \) is a 
 multiple of \( n \).

 We first prove that this relation is an equivalence relation.

 Let \( a \in  S \) then, \( a - a = 0\). Hence, \( a \sim a \). Therefore, this relation
 is reflexive.

 Let \( a, b \in S \) and \( a \sim  b \). Then \( a - b = k\) where \( k  \) is a multiple
 of \( n \). Hence, \( b - a = - k\) and \( -k \) is also a multiple of \( n \).  
 Therefore, \( b \sim a   \). Hence, this relation is symmetric. 

 Let \( a,b and c \in  S\) and \( a \sim b \) and \( b \sim c \). Then we know that
 \( a - b  \) and \( b - c \) are multiples of \( n \).
 So,
 \begin{align}
   a - b &= pn \\
   b - c &= qn
 .\end{align}

 For some \( p,q \in \mathbb{Z} \). Now we consider $(1) + (2)$, 

 \begin{align*}
   a - b + b - c &= pn - qn\\ 
   a - c &= \left( p - q \right)n
 .\end{align*}

 Since, \( p - q \in \mathbb{Z} \) as \( p,q \in \mathbb{Z}  \). Therefore,
 \( a - c \) is a multiple of \( n \). Hence, \( a - c \).
 Therefore, this relation is transitive as well.

 So, this relation is, reflexive, symmetric and transitive. Hence this 
 is an equivalence relation.

 Now we show that there are only \( n \), equivalence classes for this relation. Note that,

 \begin{align*}
   cl(0) &= \{ m \times n \mid m \in \mathbb{Z}\}   \\
   cl(1) &= \{ m \times  n + 1 \mid m \in Z\}  \\
   cl(2) &= \{ m \times  n + 2 \mid m \in Z\}  \\
   \vdots \\
   cl(n) &= \{ m \times n \mid m \in \mathbb{Z}\} = cl(0)
 .\end{align*}

 Hence, by inspection we notice that the equivalence classes 
 start cycling every \( n \) terms \( n \). It is also clear that, \( \{cl(0), cl(1), \ldots, cl(n-1)\}\), are distinct equivalence classes. 
 Now to show that only \( n \) equivalence classes exist we show that \( cl(k)  \in  \{cl(0), \ldots, cl(n-1)\}\) for
 all \( k \in \mathbb{Z} \).

 Now, let \( k \in \mathbb{Z} \), the we know that, \(  mod(k,n)  = a\) where \( m \in \{0, \ldots, n-1\}   \).

 Therefore, \( cl(k) = \{ m \times n + a \mid m \in \mathbb{Z}\}  \in \{cl(0), cl(1), \ldots, cl(n-1)\} \).

 Hence, there are \(n  \) equivalence classes for this equivalence relation which are \( \{cl(0), \ldots, cl(n-1)\}   \).

 \subsection{Solution 13}

 We state theorem 1.1.1 first below, 

 \begin{theorem}
 The distinct equivalence classes of an equivalence relation on \( A \), provide us with a decomposition of \( A \) as a union of mututally disjoint subsets
 Conversely, given a decomposition of \( A \) as a union of mutually disjoint, non empty subsets, we can define an equivalence relation on \( A \) for which
 these subsets are distinct equivalence classes.
 \end{theorem}

